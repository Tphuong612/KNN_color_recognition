\documentclass[../BTL.tex]{subfiles}
\begin{document}

Chương này có độ dài không quá 10 trang. Nếu cần trình bày dài hơn, sinh viên đưa vào phần phụ lục. Chú ý đây là kiến thức đã có sẵn; SV sau khi tìm hiểu được thì phân tích và tóm tắt lại. Sinh viên không trình bày dài dòng, chi tiết. 

Sinh viên cần trình bày các nội dung bao gồm: Kiến thức nền tảng, cơ sở lý thuyết, các thuật toán, phương pháp nghiên cứu, v.v.

Lưu ý: Nội dung BTL phải có tính chất liên kết, liền mạch, và nhất quán. Vì vậy, các công nghệ/thuật toán trình bày trong chương này phải khớp với nội dung giới thiệu của sinh viên ở phần trước đó. 

Trong chương này, để tăng tính khoa học và độ tin cậy, sinh viên nên chỉ rõ nguồn kiến thức mình thu thập được ở tài liệu nào, đồng thời đưa tài liệu đó vào trong danh sách tài liệu tham khảo.

Ví dụ: Với đề tài "Thuật toán cây quyết định ứng dụng trong bài toán phân loại văn bản", SV trình bày các nội dung trong phần "Cơ sở lý thuyết" như sau:
\begin{itemize}
    \item[1.] Giới thiệu về bài toán phân loại văn bản
    \item[2.] Thuật toán cây quyết định
    \begin{itemize}
        \item[2.1] Thuật toán ID3
        \item[2.2] Information Gain
        \item[2.3] Ưu nhược điểm của thuật toán cây quyết định
    \end{itemize}
    \item[3.] Ứng dụng thuật toán cây quyết định trong bài toán phân loại văn bản.
\end{itemize}
\end{document}
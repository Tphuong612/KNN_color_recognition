\documentclass[../BTL.tex]{subfiles}
\begin{document}
\section{Kết luận}
Sinh viên so sánh kết quả nghiên cứu hoặc sản phẩm của mình với các nghiên cứu hoặc sản phẩm tương tự.

Sinh viên cần trình bày tất cả những nội dung đóng góp mà mình thấy tâm đắc nhất trong suốt quá trình làm BTL. Đó có thể là một loạt các vấn đề khó khăn mà sinh viên đã từng bước giải quyết được, là giải thuật cho một bài toán cụ thể, là giải pháp tổng quát cho một lớp bài toán, hoặc là mô hình/kiến trúc hữu hiệu nào đó được sinh viên thiết kế.

Chương này \textbf{là cơ sở quan trọng} để các thầy cô đánh giá sinh viên. Vì vậy, sinh viên cần phát huy tính sáng tạo, khả năng phân tích, phản biện, lập luận, tổng quát hóa vấn đề và tập trung viết cho thật tốt.
Mỗi giải pháp hoặc đóng góp của sinh viên cần được trình bày trong một mục độc lập bao gồm ba mục con: (i) dẫn dắt/giới thiệu về bài toán/vấn đề, (ii) giải pháp, và (iii) kết quả đạt được (nếu có).

Sinh viên lưu ý \textbf{không trình bày lặp lại nội dung}. Những nội dung đã trình bày chi tiết trong các chương trước không được trình bày lại trong chương này. Vì vậy, với nội dung hay, mang tính đóng góp/giải pháp, sinh viên chỉ nên tóm lược/mô tả sơ bộ trong các chương trước, đồng thời tạo tham chiếu chéo tới đề mục tương ứng trong Chương 4 này. Chi tiết thông tin về đóng góp/giải pháp được trình bày trong mục đó.

Sinh viên phân tích trong suốt quá trình thực hiện BTL, mình đã làm được gì, chưa làm được gì, các đóng góp nổi bật là gì, và tổng hợp những bài học kinh nghiệm rút ra nếu có.

\section{Hướng phát triển}
Trong phần này, sinh viên trình bày định hướng công việc trong tương lai để hoàn thiện sản phẩm hoặc nghiên cứu của mình.

Trước tiên, sinh viên trình bày các công việc cần thiết để hoàn thiện các chức năng/nhiệm vụ đã làm. Sau đó sinh viên phân tích các hướng đi mới cho phép cải thiện và nâng cấp các chức năng/nhiệm vụ đã làm.
\end{document}